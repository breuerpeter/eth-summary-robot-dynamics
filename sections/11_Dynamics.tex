\section{DYNAMICS}

\begin{whitebox}{\textbf{DYNAMICS}}
    \begin{itemize}
        \item Description of the cause of motion
        \item Input: torque/force $\bm{\tau}$ acting on system
        \item Output: motion $\Ddot{\bm{q}}$ of system
        \item Methods deriving equations of motion (EOM) based on principle of virtual work
        \begin{itemize}
            \item Newton-Euler
            \item Projected Newton-Euler
            \item Lagrange II
        \end{itemize}
    \end{itemize}
\end{whitebox}

\begin{yellowbox}{\textbf{NOMENCLATURE}}
    \begin{tabularx}{\columnwidth}{ll}
        $dm$ & Infinitesimal mass element on body $\mathcal{B}$\\
        \addlinespace[2pt]
        $M$ & Point of particle $dm$\\
        \addlinespace[2pt]
        $S$ & Center of gravity (COG) and origin of $\mathcal{B}$\\
        \addlinespace[2pt]
        $\mathcal{B}$ & Body containing particles $dm$\\
        \addlinespace[2pt]
        $d\bm{F}_\mathrm{ext}$ & Resultant external force acting on $dm$\\
        \addlinespace[2pt]
        $\delta W$ & Virtual work\\
        \addlinespace[2pt]
        $\delta(\cdot)$ & Virtual quantity\\
        \addlinespace[2pt]
        %$\delta\bm{r}$ & Virtual displacement of particle $dm$\\
        %$\delta\bm{r}_S$ & Virtual displacement of $S$\\
        %$\delta\bm{\Phi}$ & Virtual angular velocity\\
        $\bm{v}_M:=\ _\mathcal{I}\dot{\bm{r}}_{IM}$ & Absolute velocity of particle $dm$\\
        \addlinespace[2pt]
        $\bm{a}_M:=\ _\mathcal{I}\ddot{\bm{r}}_{IM}$ & Absolute acceleration of particle $dm$\\
        \addlinespace[2pt]
        $\bm{\rho}:=\ _\mathcal{I}\bm{r}_{SM}$ & Vector from $S$ to a particle $dm$\\
        \addlinespace[2pt]
        $\bm{F}_\mathrm{ext}$ & Resultant external force on $S$\\
        \addlinespace[2pt]
        $\bm{T}_\mathrm{ext}$ & Resultant external torque on $\mathcal{B}$ w.r.t $S$\\
        \addlinespace[2pt]
        $\bm{p}_S$ & Linear momentum/impulse of $S$\\
        \addlinespace[2pt]
        $\bm{N}_S$ & Angular momentum of $S$\\
        \addlinespace[2pt]
        $\bm{v}_S:=\ _\mathcal{I}\dot{\bm{r}}_{\mathcal{I}S}$ & Absolute velocity of $S$\\
        \addlinespace[2pt]
        $\bm{a}_S:=\ _\mathcal{I}\ddot{\bm{r}}_{\mathcal{I}S}$ & Absolute acceleration of $S$\\
        \addlinespace[2pt]
        $\bm{I}_S$ & Inertial tensor of $\mathcal{B}$ w.r.t. $S$\\
        \addlinespace[2pt]
        $\bm{\Omega}_\mathcal{B}:=\ _\mathcal{I}\ddot{\bm{r}}_{\mathcal{I}S}$ & Absolute angular velocity of $\mathcal{B}$\\
        \addlinespace[2pt]
        $\bm{\Psi}_\mathcal{B}:=\ \dot{\bm{\Omega}}_\mathcal{B}$ & Absolute angular acceleration of $\mathcal{B}$\\
        \addlinespace[2pt]
        $n_q$ & Number of generalized coordinates\\
        \addlinespace[2pt]
        $n_c$ & Number of contact forces\\
        \addlinespace[2pt]
        $n_b$ & Number of bodies in multi-body system\\
        \addlinespace[2pt]
        $\bm{M}(\bm{q})\in\mathbb{R}^{n_q\times n_q}$ & Generalized mass matrix\\
        \addlinespace[2pt]
        $\bm{b}(\bm{q},\dot{\bm{q}})\in\mathbb{R}^{n_q}$ & Coriolis and centrifugal terms\\
        \addlinespace[2pt]
        $\bm{g}(\bm{q})\in\mathbb{R}^{n_q}$ & Gravitational terms\\
        \addlinespace[2pt]
        $\bm{\tau}\in\mathbb{R}^{n_q}$ & \makecell[l]{Forces/torques acting in direction\\
        of generalized coordinates \\ (all $\bm{q}$ assumed actuated)} \\
        \addlinespace[2pt]
        $\bm{F_\mathrm{ext}}\in\mathbb{R}^{3n_c}$ & \makecell[l]{External forces acting on system\\ (e.g. contacts)}\\
        \addlinespace[2pt]
        $\bm{J_\mathrm{ext}}(\bm{q})\in\mathbb{R}^{3n_c\times n_q}$ & Position Jacobian of external forces\\
        \addlinespace[2pt]
        $\mathcal{L}$ & Lagrangian\\
        \addlinespace[2pt]
        $\mathcal{T}$ & Kinetic energy\\
        \addlinespace[2pt]
        $\mathcal{U}$ & Potential energy

    \end{tabularx}
\end{yellowbox}

\begin{whitebox}{\textbf{BODY PROPERTIES}}
    \begin{itemize}
        \item Body mass
        \begin{align*}
            m:=\int_\mathcal{B}dm
        \end{align*}
        \item Center of mass/gravity
        \begin{align*}
            \bm{0}:=\int_\mathcal{B}\bm{\rho}dm
        \end{align*}
        \item Inertia matrix/inertial tensor around COG
        \begin{align*}
            \bm{I}_S:=\int_{\mathcal{B}}-[\bm{\rho}]_{\times}^2 \mathrm{~d} m=\int_{\mathcal{B}}[\bm{\rho}]_{\times}[\bm{\rho}]_{\times}^\top \mathrm{~d} m
        \end{align*}
    \end{itemize}
\end{whitebox}

\vfill\null 
\columnbreak

\begin{whitebox}{\textbf{PRINCIPLE OF VIRTUAL WORK (D'ALEMBERT)}}
    \begin{itemize}
        \item Dynamic equilibrium imposes zero virtual work
        \mathbox{
        \delta W=\int_\mathcal{B}\delta _\mathcal{I}\bm{r}_{IM}^\top\cdot\left(_\mathcal{I}\ddot{\bm{r}}_{IM}dm-d\bm{F}_\mathrm{ext}\right)=0
        }
        \faWarning\ Looking at a single rigid body
        \item Variational notation with $\delta$ describes, for a fixed instance in time, all possible directions the quantity may move while satisfying applicable constraints
        \item Quantities
        \begin{itemize}
            \item Absolute velocity
            \begin{align*}
                _\mathcal{I}\bm{r}_{IM}&=\ _\mathcal{I}\bm{r}_{IS}+\ _\mathcal{I}\bm{r}_{SM}\\
                _\mathcal{I}\dot{\bm{r}}_{IM}&=\ _\mathcal{I}\dot{\bm{r}}_{IS}+\ _\mathcal{I}\bm{\omega}_\mathcal{IB}\times\ _\mathcal{I}\bm{r}_{SM}\\
                \Longleftrightarrow\bm{v}_M&=\bm{v}_S+\bm{\Omega}_\mathcal{B}\times\bm{\rho}\\
                &=
                \begin{bmatrix}
                    \mathbb{I}_{3\times3} & -
                    \begin{bmatrix}
                        \bm{\rho}
                    \end{bmatrix}_\times
                \end{bmatrix}
                \begin{bmatrix}
                    \bm{v}_S\\
                    \bm{\Omega}_\mathcal{B}
                \end{bmatrix}
            \end{align*}
            \item Absolute acceleration
            \begin{align*}
                _\mathcal{I}\ddot{\bm{r}}_{IM}=\bm{a}_M&=\bm{a}_S+\bm{\Psi}_\mathcal{B}\times\bm{\rho}+\bm{\Omega}_\mathcal{B}\times(\bm{\Omega}_\mathcal{B}\times\bm{\rho})\\
                &=
                \begin{bmatrix}
                    \mathbb{I}_{3\times3} & -
                    \begin{bmatrix}
                        \bm{\rho}
                    \end{bmatrix}_\times
                \end{bmatrix}
                \begin{bmatrix}
                    \bm{a}_S\\
                    \bm{\Psi}_\mathcal{B}
                \end{bmatrix}+
                \begin{bmatrix}
                    \bm{\Omega}_\mathcal{B}
                \end{bmatrix}_\times
                \begin{bmatrix}
                    \bm{\Omega}_\mathcal{B}
                \end{bmatrix}_\times\bm{\rho}
            \end{align*}
            \item Virtual displacement
            \begin{align*}
                \delta _\mathcal{I}\bm{r}_{IM}&=\delta _\mathcal{I}\bm{r}_{IS}+\delta_\mathcal{I}\bm{\omega}_\mathcal{IB}\times\bm{\rho}
                =
                \begin{bmatrix}
                    \mathbb{I}_{3\times3} & -
                    \begin{bmatrix}
                        \bm{\rho}
                    \end{bmatrix}_\times
                \end{bmatrix}
                \begin{bmatrix}
                \delta_\mathcal{I}\bm{r}_{IS}\\
                \delta_\mathcal{I}\bm{\omega}_\mathcal{IB}
                \end{bmatrix}
            \end{align*}
        \end{itemize}
        \item Plug expressions above into D'Alembert's Principle and use body properties as well as $\int_\mathcal{B}\begin{bmatrix}\bm{\rho}\end{bmatrix}_\times d\bm{F}_\mathrm{ext}=\bm{T}_\mathrm{ext}$, etc.
        \begin{center}
            \resizebox{0.90\textwidth}{!}{$
            \begin{aligned}
                \delta W=
                \begin{bmatrix}
                    \delta_\mathcal{I}\bm{r}_{IS}\\
                    \delta_\mathcal{I}\bm{\omega}_\mathcal{IB}
                \end{bmatrix}^\top\left(
                \begin{bmatrix}
                    \mathbb{I}_{3\times3}m & \bm{0}\\
                    \bm{0} & \bm{I}_S
                \end{bmatrix}
                \begin{bmatrix}
                    \bm{a}_S\\
                    \bm{\Psi}_\mathcal{B}
                \end{bmatrix}+
                \begin{bmatrix}
                    \bm{0}\\
                    \begin{bmatrix}
                        \bm{\Omega}_\mathcal{B}
                    \end{bmatrix}_\times\bm{I}_S\bm{\Omega}_\mathcal{B}
                \end{bmatrix}-
                \begin{bmatrix}
                    \bm{F}_\mathrm{ext}\\
                    \bm{T}_\mathrm{ext}
                \end{bmatrix}\right)=0
            \end{aligned}$}    
        \end{center}
        \item Use the definitions
        \begin{align*}
            \bm{p}_S&=m\bm{v}_S\\
            \bm{N}_S&=\bm{I}_S\bm{\Omega}_\mathcal{B}\\
            \dot{\bm{p}}_S&=m\bm{a}_S\\
            \dot{\bm{N}}_S&=\bm{I}_S\bm{\Psi}_\mathcal{B}+\bm{\Omega}_\mathcal{B}\times\bm{I}_S\bm{\Omega}_\mathcal{B}\\
        \end{align*}
        \begin{align*}
             \begin{bmatrix}
                \delta_\mathcal{I}\bm{r}_{IS}\\
                \delta_\mathcal{I}\bm{\omega}_\mathcal{IB}
            \end{bmatrix}^\top
            \left(
            \begin{bmatrix}
                \dot{\bm{p}}_S\\
                \dot{\bm{N}}_S
            \end{bmatrix}-
            \begin{bmatrix}
                \bm{F}_\mathrm{ext}\\
                \bm{T}_\mathrm{ext}
            \end{bmatrix}
            \right)=0\quad\forall
            \begin{bmatrix}
                \delta_\mathcal{I}\bm{r}_{IS}\\
                \delta_\mathcal{I}\bm{\omega}_\mathcal{IB}
            \end{bmatrix}_\mathrm{consistent}       
        \end{align*}
        \item For a single, free, rigid rigid body (can move in all directions)
        \mathbox{
        \dot{\bm{p}}_S&=\bm{F}_\mathrm{ext}\\
        \dot{\bm{N}}_S&=\bm{T}_\mathrm{ext}
        }
    \end{itemize}
\end{whitebox}

\begin{whitebox}{\textbf{EULER'S LAWS OF MOTION}}
    \begin{itemize}
        \item Conservation of linear and angular momentum
        \mathbox{
        &\dot{\bm{p}}_S=\sfrac{d}{dt}\left(m\bm{v}_S\right)=m\bm{a}_S=\bm{F}_\mathrm{ext}\\
        &\dot{\bm{N}}_S=\sfrac{d}{dt}\left(\bm{I}_S\bm{\Omega}_\mathcal{B}\right)=\bm{I}_S\bm{\Psi}_\mathcal{B}+\bm{\Omega}_\mathcal{B}\times\bm{I}_S\bm{\Omega}_\mathcal{B}=\bm{T}_\mathrm{ext}
        }
        \faWarning\ Only valid for free rigid bodies (and w.r.t. inertial frame)
    \end{itemize}
\end{whitebox}

\begin{whitebox}{\textbf{FIXED BASE DYNAMICS}}
    \mathbox{
    \bm{M}(\bm{q}) \ddot{\bm{q}}+\bm{b}(\bm{q}, \dot{\bm{q}})+\bm{g}(\bm{q})=\bm{\tau}+\bm{J}_\mathrm{ext}(\bm{q})^\top \bm{F}_\mathrm{ext}
    }
\end{whitebox}

\begin{whitebox}{\textbf{NEWTON-EULER METHOD}}
    \begin{itemize}
        \item Pros and cons
        \begin{itemize}
            \itemPro Intuitively clear and direct access to constraining forces
            \itemCon Huge combinatorial problem with many bodies
        \end{itemize}
        \item Idea
        \begin{enumerate}
            \item Free body diagrams of all bodies
            \item Introduce constraining force at body interfaces
            \item Apply Euler's laws of motion to individual bodies
            \item Eliminate the constraining forces
        \end{enumerate}
        \item Example: 3D manipulator with $n_j$ 1-DOF joints
        \begin{itemize}
            \item $n_j$ moving links with 1 DOF $\implies n_j$ generalized coordinates and $5n_j$ constraining forces/torques
        \end{itemize}
    \end{itemize}
\end{whitebox}

\begin{whitebox}{\textbf{PROJECTED NEWTON-EULER METHOD}}
    \begin{itemize}
        \item Apply principle of virtual work to a multi-body system (body frames $\mathcal{B}_i$ with COGs $S_i$, $i=1,\dots,n_b$)
        \begin{center}
            \resizebox{0.90\textwidth}{!}{$
            \begin{aligned}
                \sum_{i=1}^{n_b}
                \begin{bmatrix}
                    \delta_\mathcal{I}\bm{r}_{IS_i}\\
                    \delta_\mathcal{I}\bm{\omega}_{\mathcal{IB}_i}
                \end{bmatrix}^\top
                \left(
                \begin{bmatrix}
                    \Dot{\bm{p}}_{S_i}\\
                    \Dot{\bm{N}}_{S_i}
                \end{bmatrix}-
                \begin{bmatrix}
                    \bm{F}_{\mathrm{ext},i}\\
                    \bm{T}_{\mathrm{ext},i}
                \end{bmatrix}
                \right)=0\quad\forall
                \begin{bmatrix}
                    \delta_\mathcal{I}\bm{r}_{IS_i}\\
                    \delta_\mathcal{I}\bm{\omega}_{\mathcal{IB}_i}
                \end{bmatrix}_\mathrm{consistent}
            \end{aligned}$}    
        \end{center}
        \item Express change of impulse and angular momentum in generalized coordinates
        \begin{itemize}
            \item Twist and its time derivative
            \begin{align*}
                _\mathcal{I}\bm{w}_{S_i}&=
                \begin{bmatrix}
                    \bm{v}_{S_i}\\
                    \bm{\Omega}_\mathcal{B}
                \end{bmatrix}=
                \begin{bmatrix}
                    _\mathcal{I}\bm{J}_{0_P,S_i}\\
                    _\mathcal{I}\bm{J}_{0_R,S_i}
                \end{bmatrix}\dot{\bm{q}}\\
                _\mathcal{I}\dot{\bm{w}}_{S_i}&=
                \begin{bmatrix}
                    \bm{a}_{S_i}\\
                    \bm{\Psi}_\mathcal{B}
                \end{bmatrix}=
                \begin{bmatrix}
                    _\mathcal{I}\bm{J}_{0_P,S_i}\\
                    _\mathcal{I}\bm{J}_{0_R,S_i}
                \end{bmatrix}\ddot{\bm{q}}+
                \begin{bmatrix}
                    _\mathcal{I}\dot{\bm{J}}_{0_P,S_i}\\
                    _\mathcal{I}\dot{\bm{J}}_{0_R,S_i}
                \end{bmatrix}\dot{\bm{q}}
            \end{align*}
            where $_\mathcal{I}\bm{J}_{0,S_i}$ is the geometric Jacobian of $S_i$
            \item Impulse and angular momentum time derivatives
            \begin{flushleft}
                \resizebox{0.90\textwidth}{!}{$
                \begin{aligned}
                    \begin{bmatrix}
                        \dot{\bm{p}}_{S_i}\\
                        \dot{\bm{N}}_{S_i}
                    \end{bmatrix}&=
                    \begin{bmatrix}
                        m_i\bm{a}_{S_i}\\
                        \bm{I}_{S_i}\bm{\Psi}_{\mathcal{B}_i}+\bm{\Omega}_{\mathcal{B}_i}\times\bm{I}_{S_i}\bm{\Omega}_{\mathcal{B}_i}\\
                    \end{bmatrix}\\
                    &=\begin{bmatrix}
                        m_i\ _\mathcal{I}\bm{J}_{0_P,S_i}\\
                        \bm{I}_{S_i}\ _\mathcal{I}\bm{J}_{0_R,S_i}
                    \end{bmatrix}\ddot{\bm{q}}+
                    \begin{bmatrix}
                        m_i\ _\mathcal{I}\dot{\bm{J}}_{0_P,S_i}\dot{\bm{q}}\\
                        \bm{I}_{S_i}\ _\mathcal{I}\dot{\bm{J}}_{0_R,S_i}\dot{\bm{q}}+\ _\mathcal{I}\bm{J}_{0_R,S_i}\dot{\bm{q}}\times\bm{I}_{S_i}\ _\mathcal{I}\bm{J}_{0_R,S_i}\dot{\bm{q}}
                    \end{bmatrix}
                \end{aligned}$}    
            \end{flushleft}
        \end{itemize}
        \item Express virtual displacements in generalized coordinates
        \begin{align*}
            \begin{bmatrix}
                \delta_\mathcal{I}\bm{r}_{IS_i}\\
                \delta_\mathcal{I}\bm{\omega}_{\mathcal{IB}_i}
            \end{bmatrix}=
            \begin{bmatrix}
                _\mathcal{I}\bm{J}_{0_P,S_i}\\
                _\mathcal{I}\bm{J}_{0_R,S_i}
            \end{bmatrix}\delta\bm{q}
        \end{align*}
        \item Plug above expressions into principle of virtual work for a multi-body system
        \begin{small}
            \begin{align*}
                &\delta\bm{q}^\top\Big(\Big.\sum_{i=1}^{n_b}\underbrace{
                \begin{bmatrix}
                    _\mathcal{I}\bm{J}_{0_P,S_i}\\
                    _\mathcal{I}\bm{J}_{0_R,S_i}
                \end{bmatrix}^\top
                \begin{bmatrix}
                        m_i\ _\mathcal{I}\bm{J}_{0_P,S_i}\\
                        \bm{I}_{S_i}\ _\mathcal{I}\bm{J}_{0_R,S_i}
                \end{bmatrix}}_{\bm{M}(\bm{q})}\ddot{\bm{q}}+\hdots\\
                &\hdots+\underbrace{
                \begin{bmatrix}
                    _\mathcal{I}\bm{J}_{0_P,S_i}\\
                    _\mathcal{I}\bm{J}_{0_R,S_i}
                \end{bmatrix}^\top
                \begin{bmatrix}
                    m_i\ _\mathcal{I}\dot{\bm{J}}_{0_P,S_i}\dot{\bm{q}}\\
                    \bm{I}_{S_i}\ _\mathcal{I}\dot{\bm{J}}_{0_R,S_i}\dot{\bm{q}}+\ _\mathcal{I}\bm{J}_{0_R,S_i}\dot{\bm{q}}\times\bm{I}_{S_i}\ _\mathcal{I}\bm{J}_{0_R,S_i}\dot{\bm{q}}
                \end{bmatrix}}_{\bm{b}(\bm{q},\dot{\bm{q}})}-\hdots\\
                &\hdots-\underbrace{
                \begin{bmatrix}
                    _\mathcal{I}\bm{J}_{0_P,S_i}\\
                    _\mathcal{I}\bm{J}_{0_R,S_i}
                \end{bmatrix}^\top
                \begin{bmatrix}
                    \bm{F}_{\mathrm{ext},i}\\
                    \bm{T}_{\mathrm{ext},i}
                \end{bmatrix}}_{\bm{g}(\bm{q})}
                \Big.\Big)=0\quad\forall\delta\bm{q}_\mathrm{consistent}
            \end{align*}
        \end{small}
        \item Extract matrices for the equations of motion
        \begin{itemize}
            \item Mass matrix
            \begin{align*}
                \bm{M}(\bm{q})=\sum_{i=1}^{n_b}\left(_{\color{red}\mathcal{A}}\bm{J}_{0_P,S_i}^\top m_i\ _{\color{red}\mathcal{A}}\bm{J}_{0_P,S_i}+\ _{\color{blue}\mathcal{B}}\bm{J}_{0_R,S_i}^\top\  _{\color{blue}\mathcal{B}}\bm{I}_{S_i}\ _{\color{blue}\mathcal{B}}\bm{J}_{0_R,S_i}\right)
            \end{align*}
            \item Coriolis and centrifugal terms
            \begin{align*}
                &\bm{b}(\bm{q},\dot{\bm{q}})=\sum_{i=1}^{n_b}\Big(\Big.\ _{\color{red}\mathcal{A}}\bm{J}_{0_P,S_i}^\top m_i\ _{\color{red}\mathcal{A}}\dot{\bm{J}}_{0_P,S_i}\dot{\bm{q}}+\ _{\color{blue}\mathcal{B}}\bm{J}_{0_R,S_i}^\top\cdot\hdots\\
                &\hdots\cdot\left(_{\color{blue}\mathcal{B}}\bm{I}_{S_i}\ _{\color{blue}\mathcal{B}}\dot{\bm{J}}_{0_R,S_i}\dot{\bm{q}}+\ _{\color{blue}\mathcal{B}}\bm{\Omega}_{\mathcal{B}_i}\times\ _{\color{blue}\mathcal{B}}\bm{I}_{S_i}\ _{\color{blue}\mathcal{B}}\bm{\Omega}_{\mathcal{B}_i}\right)\Big.\Big)
            \end{align*}
            \item Gravitational terms
            % TODO: this is not the same as underbraced expression?!
            % TODO: what is F_gi?
            \begin{align*}
                \bm{g}(\bm{q})=\sum_{i=1}^{n_b}\left(-\ _{\color{red}\mathcal{A}}\bm{J}_{0_P,S_i}^\top\ _{\color{red}\mathcal{A}}\bm{F}_{g_i}\right)
            \end{align*}
        \end{itemize}
        \faWarning\ $\bm{M},\bm{b},\bm{g}$ are in "generalized space" (multiplied with $\bm{q}$) so summation terms can be in different frames ${\color{red}\mathcal{A}},\color{blue}\mathcal{B}$)
    \end{itemize}
\end{whitebox}

\begin{whitebox}{\textbf{LAGRANGE II METHOD}}
    \begin{itemize}
        \item Lagrangian $\mathcal{L}(\bm{q},\dot{\bm{q}})=\mathcal{T}(\bm{q},\dot{\bm{q}})-\mathcal{U}(\bm{q})$
        \item Lagrangian equation
        \mathbox{
        \frac{d}{dt}\left(\frac{\partial\mathcal{L}}{\partial\dot{\bm{q}}}\right)^\top-\left(\frac{\partial\mathcal{L}}{\partial\bm{q}}\right)^\top=\bm{\tau}\\
        \Longleftrightarrow\frac{d}{dt}\left(\frac{\partial\mathcal{T}}{\partial\dot{\bm{q}}}\right)^\top-\left(\frac{\partial\mathcal{T}}{\partial\bm{q}}\right)^\top+\left(\frac{\partial\mathcal{U}}{\partial\bm{q}}\right)^\top=\bm{\tau}
        }
        \item Kinetic energy
        \begin{align*}
            \small
            T&=\sum_{i=1}^{n_b}\left(\frac{1}{2}m_i\ _{\color{red}\mathcal{A}}\dot{\bm{r}}_{{\color{red}A}S_i}^\top\ _{\color{red}\mathcal{A}}\dot{\bm{r}}_{{\color{red}A}S_i}+\frac{1}{2}\ _{\color{blue}\mathcal{B}}\bm{\Omega}_{\mathcal{B}_i}^\top\ _{\color{blue}\mathcal{B}}\bm{I}_{S_i}\ _{\color{blue}\mathcal{B}}\bm{\Omega}_{\mathcal{B}_i}\right)\\
            &=\frac{1}{2}\dot{\bm{q}}^\top\sum_{i=1}^{n_b}\left(_{\color{red}\mathcal{A}}\bm{J}_{0_P,S_i}^\top m_i\ _{\color{red}\mathcal{A}}\bm{J}_{0_P,S_i}+\ _{\color{blue}\mathcal{B}}\bm{J}_{0_R,S_i}^\top\  _{\color{blue}\mathcal{B}}\bm{I}_{S_i}\ _{\color{blue}\mathcal{B}}\bm{J}_{0_R,S_i}\right)\dot{\bm{q}}\\
            &=\frac{1}{2}\dot{\bm{q}}^\top\bm{M}(\bm{q})\dot{\bm{q}}
        \end{align*}
        \item Potential energy
        \begin{itemize}
            \item Gravitational forces
            \mathbox{
            &_\mathcal{I}\bm{F}_{g_i}=m_ig\ _\mathcal{I}\bm{e}_g\\
            &\mathcal{U}_g=-\sum_{i=1}^{n_b}\ _\mathcal{I}\bm{r}_{IS_i}^\top\  _\mathcal{I}\bm{F}_{g_i}
            }
            \item Spring forces
            \begin{minipage}[c]{0.7\linewidth}
                \mathbox{
                &\bm{F}_E=k_j\left(||\bm{r}-\bm{r}_0||-d_0\right)\frac{\bm{r}-\bm{r}_0}{||\bm{r}-\bm{r}_0||}\\
                &\mathcal{U}_{E_j}=\frac{1}{2}k_j\left(d(\bm{q})-d_0\right)^2
                }
            \end{minipage}%
            \begin{minipage}[c]{0.3\linewidth}
                \includegraphics[width=\textwidth]{media/Spring_Force.drawio.pdf}
            \end{minipage}
        \end{itemize}
    \end{itemize}
\end{whitebox}

\begin{whitebox}{\textbf{EXTERNAL FORCES AND TORQUES}}
    \begin{itemize}
        \item Generalized force $\bm{\tau}$ (represented in the space of generalized coordinates) can have contributions from:
        \begin{itemize}
            \item External forces $\bm{F}_{\mathrm{ext},j}$ acting in points $P_j$
            \begin{align*}
                \bm{\tau}_{F,\mathrm{ext}}=\sum_{j=1}^{n_{f,\mathrm{ext}}}\ _\mathcal{A}\bm{J}_{0_P,P_j}^\top\ _\mathcal{A}\bm{F}_{\mathrm{ext},j}\
            \end{align*}
            \item External torques $\bm{T}_{\mathrm{ext},k}$ acting on bodies $\mathcal{B}_k$
            \begin{align*}
                \bm{\tau}_{T,\mathrm{ext}}=\sum_{k=1}^{n_{m,\mathrm{ext}}}\ _\mathcal{A}\bm{J}_{0_R,\mathcal{B}_k}^\top\ _\mathcal{A}\bm{T}_{\mathrm{ext},k}
            \end{align*}
            \item Combined
            \begin{align*}
                \bm{\tau}_\mathrm{ext}=\bm{\tau}_{F,\mathrm{ext}}+\bm{\tau}_{T,\mathrm{ext}}
            \end{align*}
            \item Actuators
            \begin{itemize}
                \item Actuator acting between body links $\mathcal{B}_{k-1}$ and $\mathcal{B}_k$ (in points $B_{k-1}$ and $B_k$, respectively) imposes a force $\bm{F}_{a,k}$ and/or torque $\bm{T}_{a,k}$ on both links equally and in opposite directions
            \end{itemize}
            \begin{align*}
                \bm{\tau}_{a,k}=(_\mathcal{A}\bm{J}_{0_P,B_k}-\ _\mathcal{A}\bm{J}_{0_P,B_{k-1}})^\top\ _\mathcal{A}\bm{F}_{a,k}+\dots\\
                \dots+(_\mathcal{A}\bm{J}_{0_R,\mathcal{B}_k}-\ _\mathcal{A}\bm{J}_{0_R,\mathcal{B}_{k-1}})^\top\ _\mathcal{A}\bm{T}_{a,k}
            \end{align*}
            \item Total
            \begin{align*}
                \bm{\tau}=\bm{\tau}_\mathrm{ext}+\sum_{k=1}^{n_A}\bm{\tau}_{a,k}
            \end{align*}
        \end{itemize}
    \end{itemize}
\end{whitebox}
