\section{QUADCOPTER MODELING}

\begin{yellowbox}{\textbf{NOMENCLATURE}}
    \begin{tabularx}{\columnwidth}{ll}
        $\mathcal{E}$ & Inertial (Earth-fixed) frame with origin $E$\\
        \addlinespace[2pt]
        $B$ & Center of gravity (COG) of $\mathcal{B}$\\
        \addlinespace[2pt]
        $\mathcal{B}$ & Body frame with origin $B$\\
        \addlinespace[2pt]
        $\phi,\theta,\psi$ & Roll, pitch and yaw angles\\
        \addlinespace[2pt]
        $_\mathcal{B}\bm{\omega}_\mathcal{EB}$ & Angular velocity of $\mathcal{B}$ (body angular rates) w.r.t. $\mathcal{E}$\\
        & (expressed in $\mathcal{B}$) with components $p,q,r$\\
        \addlinespace[2pt]
        $ _\mathcal{B}\dot{\bm{r}}_{EB}$ & Body velocity\\ % TODO: check if u,v,w are actually w.r.t. body frame
        \addlinespace[2pt]
        & (expressed in $\mathcal{B}$) with components $u,v,w$\\
        \addlinespace[2pt]
        $T_i$ & Propeller thrust forces\\
        \addlinespace[2pt]
        $Q_i$ & Propeller drag torques\\
        \addlinespace[2pt]
        $\omega_{p,i}$ & Propeller angular rates\\
        \addlinespace[2pt]
        $l$ & Arm length\\
        \addlinespace[2pt]
        $h$ & Propeller height (from COG to propeller plane)\\
        \addlinespace[2pt]
        $m$ & Total mass\\
        \addlinespace[2pt]
        $\bm{I}$ & Inertia tensor of $\mathcal{B}$ w.r.t. $B$\\
        \addlinespace[2pt]
        $\bm{F}_\mathrm{Aero}$ & Resultant aerodynamic force on $B$\\
        \addlinespace[2pt]
        $\bm{F}$ & Resultant force on $B$\\
        \addlinespace[2pt]
        $\bm{M}_\mathrm{Aero}$ & Resultant aerodynamic torque on $B$\\
        \addlinespace[2pt]
        $\bm{M}$ & Resultant moment/torque on $\mathcal{B}$ w.r.t. $B$\\
        \addlinespace[2pt]
        $b$ & Propeller thrust constant\\
        \addlinespace[2pt]
        $d$ & Propeller drag constant
        
    \end{tabularx}
\end{yellowbox}

\begin{center}
    \resizebox{0.8\linewidth}{!}{
    \includegraphics[]{media/Quadcopter_Diagram.png}
    }   
\end{center}


\begin{whitebox}{\textbf{MODELING PURPOSES}}
    \begin{itemize}
        \item System analysis: model allows evaluating characteristics (stability, controllability, power consumption, etc.) of future aircraft in flight or its behavior in various conditions
        \item Control law design and simulation: model allows comparing various control techniques and tune their parameters
    \end{itemize}
\end{whitebox}

\begin{whitebox}{\textbf{MODELING ASSUMPTIONS}}
    \begin{itemize}
        \item CoG and body frame origin coincide
        \item No interaction with ground or other surfaces
        \item Rigid and symmetric structure
        \item Rigid propellers
        \item No fuselage drag
        \item Frame (body) is symmetric in $xz$- and $yz$-plane (diagonal inertia tensor)
    \end{itemize}
\end{whitebox}

\begin{whitebox}{\textbf{MODEL COMPONENTS}}
    \tikzstyle{block} = [draw, rectangle, minimum height=3em, minimum width=6em]
    \tikzstyle{input} = [coordinate]
    \tikzstyle{output} = [coordinate]
    \tikzstyle{pinstyle} = [pin edge={to-,thin,black}]
    
    \scalebox{0.63}{
    \begin{tikzpicture}[auto, node distance=2cm,>=latex']
        % nodes
        \node [input, name=in] {};
        \node[block, right of=in, align=center, font=\small, text width=1.5cm] (motor)
        {\textbf{Motor\\Dynamics}};
        \node[block, align=center, right of=motor, font=\small, node distance=3.5cm, text width=2.5cm] (aero)
            {\textbf{Aerodynamics}};
        \node[block, align=center, right of=aero, font=\small, node distance=3.5cm, text width=1.5cm] (dyn)
            {\textbf{Body\\Dynamics}};
        \node[block, align=center, right of=dyn, font=\small, node distance=3.5cm, text width=0.5cm] (int)
            {$\int$};
        \node [output, name=out, right of=int] {};
        % arrows
        \draw[->] (in) -- node {$\bm{\omega}_p^*$} (motor);
        \draw[->] (motor) -- node {$\bm{\omega}_p$} (aero);
        \draw[->] (aero) -- node {$\bm{F,M}$} (dyn);
        \draw[->] (dyn) -- node {$\dot{\bm{v}},\dot{\bm{\omega}}$} (int);
        \draw[->] (int) -- node {$\bm{v},\bm{\omega}$} (out);
    \end{tikzpicture}}

    \begin{itemize}
        \item System input: desired rotor speeds $\bm{\omega}_r^*$ (or the corresponding motor voltage inputs)
        \item Motor dynamics are very fast and can therefore be neglected for control design
    \end{itemize}
\end{whitebox}

\begin{whitebox}{\textbf{ATTITUDE}}
    \begin{itemize}
        \item Representation using yaw, pitch, roll Tait-Bryan angles
    \begin{align*}
            \bm{C}_\mathcal{EB}&=\bm{C}_{\mathcal{E}1}(\psi)\bm{C}_{12}(\theta)\bm{C}_{2\mathcal{B}}(\phi)\\
            &=\bm{C}_z(\psi)\bm{C}_y(\theta)\bm{C}_x(\phi)\\
            &=
            \begin{bmatrix}
                c_\theta c_\psi & c_\psi s_\phi s_\theta-c_\phi s_\psi & s_\phi s_\psi+c_\phi c_\psi s_\theta\\
                c_\theta s_\psi & c_\phi c_\psi+s_\phi s_\theta s_\psi & c_\phi s_\theta s_\psi-c_\psi s_\phi\\
                -s_\theta & c_\theta s_\phi & c_\phi c_\theta
            \end{bmatrix}
        \end{align*}
        \begin{align*}
            \bm{C}_\mathcal{BE}=\bm{C}_\mathcal{EB}^\top=
            \begin{bmatrix}
                c_\theta c_\psi & c_\theta s_\psi & -s_\theta\\
                c_\psi s_\phi s_\theta-c_\phi s_\psi & c_\phi c_\psi+s_\phi s_\theta s_\psi & c_\theta s_\phi\\
                s_\phi s_\psi+c_\phi c_\psi s_\theta & c_\phi s_\theta s_\psi-c_\psi s_\phi & c_\phi c_\theta
            \end{bmatrix}
        \end{align*}
        \item Limits
        \begin{align*}
            &\phi\in(-\pi,\pi)\\
            &\theta\in(-\sfrac{\pi}{2},\sfrac{\pi}{2})\\
            &\psi\in(-\pi,\pi)
        \end{align*}
    \end{itemize}
\end{whitebox}

\vfill\null 
\columnbreak

\begin{whitebox}{\textbf{ANGULAR VELOCITY}}
    \begin{itemize}
        \item Rotation matrix $\bm{C}_\mathcal{EB}$ from $\mathcal{B}$ to $\mathcal{E}$
            \mathbox{
                \bm{C}_\mathcal{EB}=\bm{C}_{\mathcal{E}1}(\psi)\bm{C}_{12}(\theta)\bm{C}_{2\mathcal{B}}(\phi)
            }
            \begin{enumerate}
                \item $\bm{C}_{\mathcal{E}1}$ is rotation (yaw $\psi$) around $\bm{e}_z^\mathcal{E}$ $(\bm{C}_z(\psi))$
                \item $\bm{C}_{12}$ is rotation (pitch $\theta$) around $\bm{e}_y^1$ $(\bm{C}_y(\theta))$
                \item $\bm{C}_{2\mathcal{B}}$ is rotation (roll $\phi$) around $\bm{e}_x^2$ $(\bm{C}_x(\phi))$
            \end{enumerate}
        \item ZYX Tait-Bryan angles (3-2-1 intrinsic Euler angles)
            \begin{align*}
                \bm{\Theta}=
                \begin{bmatrix}
                    \phi & \theta & \psi
                \end{bmatrix}^\top
            \end{align*}
        \item Relation to angular rates $\dot{\bm{\Theta}}$
           \begin{align*}
                \dot{\bm{\Theta}}=
                \begin{bmatrix}
                    \Dot{\phi} & \Dot{\theta} & \Dot{\psi}
                \end{bmatrix}^\top\neq\ 
                _\mathcal{B}\bm{\omega}_\mathcal{EB}=
                \begin{bmatrix}
                    p & q & r
                \end{bmatrix}^\top
            \end{align*}
            \begin{align*}
                \bm{\omega}_\mathcal{EB}=
                \bm{\omega}_{\mathcal{E}1}+
                \bm{\omega}_{12}+
                \bm{\omega}_{2\mathcal{B}}&=
                \Dot{\psi}\bm{e}_z^\mathcal{E}+\Dot{\theta}\bm{e}_y^1+\Dot{\phi}\bm{e}_x^2\\
                &=\Dot{\psi}\bm{e}_z^1+\Dot{\theta}\bm{e}_y^2+\Dot{\phi}\bm{e}_x^\mathcal{B}
            \end{align*}
            \begin{align*}
                _\mathcal{B}\bm{\omega}_\mathcal{EB}&=\Dot{\psi}\bm{C}_{\mathcal{B}1}\bm{e}_z^1+\Dot{\theta}\bm{C}_{\mathcal{B}2}\bm{e}_y^2+\Dot{\phi}\bm{e}_x^\mathcal{B}\\
                &=\begin{bmatrix}
                    \bm{e}_x^\mathcal{B} & \bm{C}_{\mathcal{B}2}\bm{e}_y^2 & \bm{C}_{\mathcal{B}1}\bm{e}_z^1
                \end{bmatrix}\dot{\bm{\Theta}}\\
                &=\begin{bmatrix}
                    \bm{e}_x^\mathcal{B} & \bm{C}_x^\top(\phi)\bm{e}_y^2 & \bm{C}_{\mathcal{B}2}\bm{C}_{21}\bm{e}_z^1
                \end{bmatrix}\dot{\bm{\Theta}}\\
                &=\begin{bmatrix}
                    \bm{e}_x^\mathcal{B} & \bm{C}_x^\top(\phi)\bm{e}_y^2 & \bm{C}_x^\top(\phi)\bm{C}_y^\top(\theta)\bm{e}_z^1
                \end{bmatrix}\dot{\bm{\Theta}}\\
            \end{align*}
            \mathbox{
            _\mathcal{B}\bm{\omega}_\mathcal{EB}=\underbrace{
            \begin{bmatrix}
                    1 & 0 & -\sin\theta\\
                    0 & \cos\phi & \sin\phi\cos\theta\\
                    0 & -\sin\phi & \cos\phi\cos\theta
                \end{bmatrix}}_{\bm{E}_{R,\mathrm{Euler,}ZYX}}\dot{\bm{\Theta}}
            }
            \begin{itemize}
                \item Linearization around hover
                \begin{align*}
                    \bm{E}_{R,\mathrm{Euler,}ZYX}\Big\rvert_{\phi=\theta=0}=\mathbb{I}_{3\times3}\Longleftrightarrow\dot{\bm{\Theta}}=\ _\mathcal{B}\bm{\omega}_\mathcal{EB}
                \end{align*}
                \item Singularity: $\det(\bm{E}_{R,\mathrm{Euler,}ZYX})=-\cos\theta$\\
                $\implies$ gimbal lock at $\theta=\pm\sfrac{\pi}{2}$
            \end{itemize}
    \end{itemize}
\end{whitebox}

\begin{whitebox}{\textbf{UAV DYNAMICS}}
    \begin{align*}
        \bm{M}(\bm{q})\ddot{\bm{q}}+\bm{b}(\bm{q},\dot{\bm{q}})+\bm{g}(\bm{q})+\bm{J}_\mathrm{ext}^\top\bm{F}_\mathrm{ext}=\bm{S}^\top\bm{\tau}_\mathrm{act}
    \end{align*}
    \begin{center}
        \begin{tabularx}{\columnwidth}{ll}
            $\bm{q}$ & Generalized coordinates\\
            $\bm{M}(\bm{q})$ & Mass matrix\\
            $\bm{b}(\bm{q},\dot{\bm{q}})$ & Centrifugal and Coriolis forces\\
            $\bm{g}(\bm{q})$ & Gravity forces\\
            $\bm{\tau}_\mathrm{act}$ & Actuation torques\\
            $\bm{S}$ & Selection matrix of actuated joints\\
            $\bm{F}_\mathrm{ext}$ & External forces (exerted by system)\\
            $\bm{J}_\mathrm{ext}$ & (Geometric) Jacobian of external forces
        \end{tabularx}
    \end{center}
\end{whitebox}

\begin{whitebox}{\textbf{EQUATION OF FORCES}}
    \begin{enumerate}
        \item Conservation of linear momentum
        \begin{align*}
            \dot{\bm{p}}_S=\sfrac{d}{dt}\left(m\bm{v}_S\right)=m\bm{a}_S=\bm{F}_\mathrm{ext}
        \end{align*}
        \item Modified notation
        \begin{align*}
            \dot{\bm{p}}_B=\sfrac{d}{dt}\left(m\dot{\bm{r}}_{EB}\right)=m\ddot{\bm{r}}_{EB}=\bm{F}
        \end{align*}
        \item Alternative formulation
        \begin{align*}
            \frac{d}{dt}\biggr\rvert_\mathcal{E}(m\dot{\bm{r}}_{EB})=\sum_j\bm{F}_j
        \end{align*}
        where subscript $\mathcal{E}$ denotes time differentiation w.r.t. the inertial frame $\mathcal{E}$ and $\dot{\bm{r}}_{EB}$ is the velocity of the COG $B$
        \item Change of frame gives %(using \cref{eq:time_deriv_frames})
        \begin{align*}
            \sum_j\bm{F}_j=\ \frac{d}{dt}\biggr\rvert_\mathcal{B}(m\dot{\bm{r}}_{EB})+\bm{\omega}_\mathcal{EB}\times(m\dot{\bm{r}}_{EB})
        \end{align*}
        \item Assume constant mass and express vectors in $\mathcal{B}$
        \begin{align*}
            \frac{1}{m}\ _\mathcal{B}\bm{F}=\ _\mathcal{B}\ddot{\bm{r}}_{EB}+\ _\mathcal{B}\bm{\omega}_\mathcal{EB}\times\ _\mathcal{B}\dot{\bm{r}}_{EB}
        \end{align*}
        \item The resultant force $\bm{F}$ is composed of aerodynamic forces and gravity
        \begin{align*}
            &\frac{1}{m}\ _\mathcal{B}\bm{F}_\mathrm{Aero}+\bm{C}_\mathcal{BE}
            \begin{bmatrix}
                0\\
                0\\
                1
            \end{bmatrix}g=
            \begin{bmatrix}
                \Dot{u}\\
                \Dot{v}\\
                \Dot{w}
            \end{bmatrix}+
            \begin{bmatrix}
                p\\
                q\\
                r
            \end{bmatrix}\times
            \begin{bmatrix}
                u\\
                v\\
                w
            \end{bmatrix}\\
            \implies
            &\frac{1}{m}\ _\mathcal{B}\bm{F}_\mathrm{Aero}+
            \begin{bmatrix}
                -\sin\theta\\
                \sin\phi\cos\theta\\
                \cos\phi\cos\theta
            \end{bmatrix}g=
            \begin{bmatrix}
                qw-rv\\
                ru-pw\\
                pv-qu
            \end{bmatrix}=
            \begin{bmatrix}
                \Dot{u}\\
                \Dot{v}\\
                \Dot{w}
            \end{bmatrix}
        \end{align*}
    \end{enumerate}
\end{whitebox}

\begin{whitebox}{\textbf{EQUATION OF MOMENTS}}
    \begin{enumerate}
        \item Conservation of angular momentum
        \begin{align*}
            \dot{\bm{N}}_S=\sfrac{d}{dt}\left(\bm{I}_S\bm{\Omega}_\mathcal{B}\right)=\bm{I}_S\bm{\Psi}_\mathcal{B}+\bm{\Omega}_\mathcal{B}\times\bm{I}_S\bm{\Omega}_\mathcal{B}=\bm{T}
        \end{align*}
        \item Modified notation
        \begin{align*}
            \dot{\bm{N}}_B=\sfrac{d}{dt}\left(\bm{I}_B\ \bm{\omega}_\mathcal{EB}\right)=\bm{I}_B\ \dot{\bm{\omega}}_\mathcal{EB}+\bm{\omega}_\mathcal{EB}\times\bm{I}_B\ \bm{\omega}_\mathcal{EB}=\bm{M}
        \end{align*}
        \item Express vectors in $\mathcal{B}$
        \begin{align*}
            \bm{I}_B\ _\mathcal{B}\dot{\bm{\omega}}_\mathcal{EB}+\ _\mathcal{B}\bm{\omega}_\mathcal{EB}\times\bm{I}_B\ _\mathcal{B}\bm{\omega}_\mathcal{EB}=\ _\mathcal{B}\bm{M}
        \end{align*}
        \item The resultant moment $M$ is composed of aerodynamic moments
        \begin{center}
            \resizebox{0.90\textwidth}{!}{$
            \begin{aligned}
                &\begin{bmatrix}
                    I_{xx} & 0 & 0\\
                    0 & I_{yy} & 0\\
                    0 & 0 & I_{zz}
                \end{bmatrix}
                \begin{bmatrix}
                    \dot{p}\\
                    \dot{q}\\
                    \dot{r}
                \end{bmatrix}+
                \begin{bmatrix}
                    p\\
                    q\\
                    r
                \end{bmatrix}\times
                \begin{bmatrix}
                    I_{xx} & 0 & 0\\
                    0 & I_{yy} & 0\\
                    0 & 0 & I_{zz}
                \end{bmatrix}
                \begin{bmatrix}
                    p\\
                    q\\
                    r
                \end{bmatrix}=\ _\mathcal{B}\bm{M}_\mathrm{Aero}\\
                \implies&
                \begin{bmatrix}
                    I_{xx}\dot{p}+qr(I_{zz}-I_{yy})\\
                    I_{yy}\dot{q}+pr(I_{xx}-I_{zz})\\
                    I_{zz}\dot{r}
                \end{bmatrix}=\ _\mathcal{B}\bm{M}_\mathrm{Aero}
            \end{aligned}$}
        \end{center}
        using $I_{yy}=I_{zz}$ (assumption)
    \end{enumerate}
\end{whitebox}

        
\begin{whitebox}{\textbf{BODY DYNAMICS}}
    \begin{itemize}
        \item Equation of forces and moments in matrix form
        \begin{align*}
            \begin{bmatrix}
                m\mathbb{I}_{3\times3} & \bm{0}\\
                \bm{0} & \bm{I}_B
            \end{bmatrix}
            \begin{bmatrix}
                _\mathcal{B}\ddot{\bm{r}}_{EB}\\
                _\mathcal{B}\dot{\bm{\omega}}_\mathcal{EB}
            \end{bmatrix}+
            \begin{bmatrix}
                _\mathcal{B}\bm{\omega}_\mathcal{EB}\times m\ _\mathcal{B}\dot{\bm{r}}_{EB}\\
                _\mathcal{B}\bm{\omega}_\mathcal{EB}\times\bm{I}_B\ _\mathcal{B}\bm{\omega}_\mathcal{EB}
            \end{bmatrix}=
            \begin{bmatrix}
                _\mathcal{B}\bm{F}\\
                _\mathcal{B}\bm{M}
            \end{bmatrix}
        \end{align*}
    \end{itemize}
\end{whitebox}

\begin{whitebox}{\textbf{AERODYNAMIC FORCES}}
    \begin{align*}
        _\mathcal{B}\bm{F}_\mathrm{Aero}&=\ _\mathcal{B}\bm{F}_\mathrm{Thrust}=
        \begin{bmatrix}
            0\\
            0\\
            -b(\omega_{p_1}^2+\omega_{p_2}^2+\omega_{p_3}^2+\omega_{p_4}^2)
        \end{bmatrix}
    \end{align*}
    \begin{itemize}
        \item Thrust forces in the shaft direction
        \begin{align*}
            _\mathcal{B}\bm{F}_\mathrm{Thrust}=\sum_{i=1}^4
            \begin{bmatrix}
                0\\
                0\\
                -T_i
            \end{bmatrix},\quad T_i=b_i\omega_{p,i}^2
        \end{align*}
        \item Additional forces (neglected during hover)
        \begin{itemize}
            \item Hub forces along the horizontal speed
            \begin{align*}
                _\mathcal{B}\bm{H}=\sum_{i=1}^4 H_i\frac{_\mathcal{B}\dot{\bm{r}}_{EB_h}}{||_\mathcal{B}\dot{\bm{r}}_{EB_h}||},\quad_\mathcal{B}\dot{\bm{r}}_{EB_h}=
                \begin{bmatrix}
                    u & v & 0
                \end{bmatrix}^\top
            \end{align*}
        \end{itemize}
    \end{itemize}
\end{whitebox}

\begin{whitebox}{\textbf{AERODYNAMIC MOMENTS}}
    \begin{center}
        \resizebox{0.95\textwidth}{!}{$
        \begin{aligned}
            _\mathcal{B}\bm{M}_\mathrm{Aero}=\ _\mathcal{B}\bm{M}_\mathrm{Thrust}+\ _\mathcal{B}\bm{M}_\mathrm{Drag}=
            \begin{bmatrix}
                lb(\omega_{p_4}^2-\omega_{p_2}^2)\\
                lb(\omega_{p_1}^2-\omega_{p_3}^2)\\
                d(-\omega_{p_1}^2+\omega_{p_2}^2-\omega_{p_3}^2+\omega_{p_4}^2)
            \end{bmatrix}
        \end{aligned}$}
    \end{center}
    \begin{itemize}
        \item Thrust induced moment from propeller rotations
        \begin{align*}
            _\mathcal{B}\bm{M}_\mathrm{Thrust}=
            \begin{bmatrix}
                l(T_4-T_2)\\
                l(T_1-T_3)\\
                0
            \end{bmatrix}
        \end{align*}
        \item Drag torques (define required motor power)
        \begin{align*}
            _\mathcal{B}\bm{M}_\mathrm{Drag}=
            \begin{bmatrix}
                0\\
                0\\
                \sum_{i=1}^4Q_i(-1)^i
            \end{bmatrix},\quad Q_i=d_i\omega_{p,i}^2
        \end{align*}
        \item Additional moments (neglected during hover)
        \begin{itemize}
            \item Inertial counter torques % TODO: check equation
            \begin{align*}
                _\mathcal{B}\bm{M}_{\mathrm{CT}_i}=\bm{I}_{\mathrm{Prop}_i}
                \begin{bmatrix}
                    0\\
                    0\\
                    \omega_{p,i}
                \end{bmatrix}
            \end{align*}
            \item Propeller gyro effect % TODO: check equation
            \begin{align*}
                _\mathcal{B}\bm{M}_{\mathrm{Gyro}_i}=
                \begin{bmatrix}
                    \bm{I}_{\mathrm{Prop}_i}\omega_{p,i}
                \end{bmatrix}\times\ \mathcal{B}\bm{\omega}_\mathcal{EB}
            \end{align*}
            \item Rolling moments % TODO: check equation
            \begin{align*}
                _\mathcal{B}\bm{R}=\sum_{i=1}^4R_i(-1)^{i-1}\frac{_\mathcal{B}\dot{\bm{r}}_{EB_h}}{||_\mathcal{B}\dot{\bm{r}}_{EB_h}||}
            \end{align*}
            \item Hub moments % TODO: check equation (and whats the p?)
            \begin{align*}
                _\mathcal{B}\bm{M}_\mathrm{Hub}=\sum_{i=1}^4H_i\ _\mathcal{B}\bm{p}_{p,i}\times\frac{_\mathcal{B}\dot{\bm{r}}_{EB_h}}{||_\mathcal{B}\dot{\bm{r}}_{EB_h}||}
            \end{align*}
        \end{itemize}
    \end{itemize}
\end{whitebox}

\begin{whitebox}{\textbf{HOVER MODEL APPROXIMATION}}
    \begin{itemize}
        \item Equations of motion in body frame
        \begin{itemize}
            \item Translational dynamics
            \mathbox{
                &m\dot{u}=m(rv-qw)-mg\sin\theta\\
                &m\dot{v}=m(pw-ru)+mg\sin\phi\cos\theta\\
                &m\dot{w}=m(qu-pv)+mg\cos\phi\cos\theta-\hdots\\
                &\quad\quad\hdots-b(\omega_{p_1}^2+\omega_{p_2}^2+\omega_{p_3}^2+\omega_{p_4}^2)
            }
            \item Rotational dynamics
            \mathbox{
            &I_{xx}\dot{p}=qr(I_{yy}-I_{zz})+lb(\omega_{p_4}^2-\omega_{p_2}^2)\\
            &I_{yy}\dot{q}=pr(I_{zz}-I_{xx})+lb(\omega_{p_1}^2-\omega_{p_3}^2)\\
            &I_{zz}\dot{r}=d(-\omega_{p_1}^2+\omega_{p_2}^2-\omega_{p_3}^2+\omega_{p_4}^2)
            }
        \end{itemize}
    \end{itemize}
\end{whitebox}

\begin{whitebox}{\textbf{PROPELLER AERODYNAMICS}}
    \begin{itemize}
        \item Propeller in hover
        \begin{itemize}
            \item Thrust force $T$ (perpendicular to propeller plane)
            \begin{align*}
                T=\frac{1}{2}\rho A_pC_T(\omega_pR_p)^2
            \end{align*}
            \item Drag torque $Q$ (torque around propeller plane; opposite to prop spin direction)
            \begin{align*}
                Q=\frac{1}{2}\rho A_pC_Q(\omega_pR_p)^2R_p
            \end{align*}
            \item $C_T,C_Q$ depend on blade pitch angle (propeller geometry), Reynolds number, etc.
        \end{itemize}
        \item Propeller in forward flight
        \begin{itemize}
            \item Hub force $H$ (perpendicular to $T$; opposes horizontal flight direction)
            \begin{align*}
                H=\frac{1}{2}\rho A_pC_H(\omega_pR_p)^2
            \end{align*}
            \item Rolling moment $R$ (around flight direction)
            \begin{align*}
                R=\frac{1}{2}\rho A_pC_R(\omega_pR_p)^2R_p
            \end{align*}
            \item $C_H,C_R$ depend on the advance ratio $\mu=\frac{||_\mathcal{B}\dot{\bm{r}}_{EB_h}||}{\omega_pR_p}$
        \end{itemize}
    \end{itemize}
\end{whitebox}

\begin{whitebox}{\textbf{MOMENTUM THEORY}}
    \begin{itemize}
        \item Idea: by actio-reactio, the power put into the fluid to change its momentum downwards is the thrust force at the propeller
    \end{itemize}
\end{whitebox}

\begin{whitebox}{\textbf{MOMENTUM THEORY ASSUMPTIONS}}
    \begin{itemize}
        \item Infinitely thin propeller disc area $A_p$
        \item Uniform thrust and velocity distribution over disc area (allows 1D flow analysis)
        \item Quasi-static airflow (constant flow properties)
        \item No viscous effects (no profile drag)
        \item Incompressible flow
        \item Stationary control volume $A$
    \end{itemize}
\end{whitebox}

\begin{whitebox}{\textbf{FLUID DYNAMICS PRINCIPLES}}
    \begin{itemize}
        \item \textbf{Conservation of fluid mass}: mass flow in and out of control volume is equal in a quasi-static flow
        \begin{align*}
            \int_A\rho(\bm{u}\cdot{\bm{n}})\ dA=0
        \end{align*}
        \begin{itemize}
            \item Fluid density $\rho$
            \item Flow speed $\bm{u}$ at surface element $dA$
            \item Surface element normal unit vector $\bm{n}$
        \end{itemize}
        \item \textbf{Conservation of fluid momentum}: net force on fluid is the change in momentum of fluid
        \begin{align*}
            \int_A\rho\bm{u}(\bm{u}\cdot\bm{n})\ dA=\cancelto{0}{-\int_Ap\bm{n}\ dA}+\bm{F}
        \end{align*}
        assuming unconstrained flow (net pressure force is $0$)
        \begin{itemize}
            \item Pressure $p$ at surface element $dA$
            \item Net force $\bm{F}$ on fluid
        \end{itemize}
        \item \textbf{Conservation of energy}: work done on fluid results in gain of kinetic energy
        \begin{align*}
            \frac{1}{2}\int_A\rho u^2(\bm{u}\cdot\bm{n})\ dA=\frac{dE}{dt}=P
        \end{align*}
        \begin{itemize}
            \item Energy $E$
            \item Power $P$
        \end{itemize}
    \end{itemize}
\end{whitebox}



\begin{whitebox}{\textbf{MOMENTUM THEORY RESULTS}}
    \begin{minipage}[c]{0.36\linewidth}
        \begin{center}
            \includegraphics[width=0.9\textwidth]{media/Momentum_Theory.drawio.pdf}
        \end{center}
    \end{minipage}%
    \begin{minipage}[c]{0.6\linewidth}
        \begin{itemize}
            \item Speed constant across propeller
            \begin{align*}
                u_1=u_2
            \end{align*}
            \item Pressure change across propeller
            \item Far wake slipstream velocity is twice the induced velocity
            \begin{align*}
                u_3=2u_1
            \end{align*}
            \item Thrust force $T=2\rho A_p(V+u_1)u_1$
            \item In hover ($V=0$)
            \begin{itemize}
                \item Thrust force $T=2\rho A_pu_1^2$
                \item Slipstream tube
                \begin{align*}
                    A_0=\infty\implies A_3=A_p/2
                \end{align*}
            \end{itemize}
        \end{itemize}
    \end{minipage}
\end{whitebox}

\begin{whitebox}{\textbf{HOVER POWER CONSUMPTION}}
    \begin{itemize}
        \item Ideal power to produce thrust
        \begin{align*}
            P=T(V+u_1)
        \end{align*}
        \begin{itemize}
            \item In hover
            \begin{align*}
                P=\frac{T^{\frac{3}{2}}}{\sqrt{2\rho A_p}},\quad T=mg
            \end{align*}
            \begin{itemize}
                \item Reducing power by decreasing disc loading $\frac{T}{A_p}$ i.e. increasing $A_p$
                \begin{itemize}
                    \item Mechanical constraint: tip Mach number
                    \item More profile/structural drag
                    \item Longer tail, etc.
                \end{itemize}
            \end{itemize}
        \end{itemize}
    \end{itemize}
\end{whitebox}

\begin{whitebox}{\textbf{PROPELLER EFFICIENCY}}
    \begin{itemize}
        \item Figure of merit
        \begin{align*}
            FM=\frac{\text{Ideal power required to hover}}{\text{Actual power required to hover}}<1
        \end{align*}
        \begin{itemize}
            \item Ideal power given by momentum theory
            \item Actual power includes profile drag, blade-tip vortex, etc.
            \item $FM$ used to compare different propellers with the same disc loading
        \end{itemize}
    \end{itemize}
\end{whitebox}

\begin{whitebox}{\textbf{BLADE ELEMENT MOMENTUM THEORY (BEMT)}}
    \begin{itemize}
        \item Divide propeller into blade elements $dr$
        \item Calculate forces for each (2D) airfoil element and sum them up
        \begin{itemize}
            \item Angle of attack and Reynolds number from relative airflow $V$ consisting of tangential velocity $\omega_pr$ and axial velocity $V_p+u_\mathrm{ind}$
            \begin{itemize}
                \item Main component $V_p$ of axial velocity calcilated using Momentum Theory
                \item $u_\mathrm{ind}$ is the induced velocity
            \end{itemize}
            \item Lift and drag polar provide lift and drag coefficient based on angle of attack
            
        \end{itemize}
    \end{itemize}
\end{whitebox}

\begin{whitebox}{\textbf{DC MOTOR MODEL}}
    \begin{itemize}
        \item Mechanical system
        \begin{itemize}
            \item Change in rotational speed depends on generated motor torque $T_m$ and drag torque $Q$
            \begin{align*}
                I_m\frac{\omega_m}{dt}=T_m(t)-Q(t)
            \end{align*}
            \item Electromagnetic field in coil generates $T_m(t)$
            \begin{align*}
                T_m(t)=k_Ti(t)
            \end{align*}
            with  torque constant $k_T$ and current $i(t)$
        \end{itemize}
        \item Electrical system
        \begin{itemize}
            \item Voltage balance
            \begin{align*}
                L\frac{d}{dt}i(t)=U(t)-Ri(t)-U_\mathrm{Ind}(t)
            \end{align*}
            with coil inductance $L$, resistance $R$, input voltage $U(t)$ and induced voltage $U_\mathrm{Ind}$ (back EMF from rotating coil inducing opposing current)
        \end{itemize}
        \item Electrical dynamics usually much faster than mechanical, therefore approximate the full (second order) system as a as first order system
    \end{itemize}
\end{whitebox}
