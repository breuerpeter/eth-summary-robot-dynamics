\section{LEGGED ROBOTICS}

\begin{whitebox}{\textbf{IDEAL ACTUATOR FOR LEGGED ROBOTICS}}
    \begin{itemize}
        \item Ideal torque source (high bandwidth, high accuracy)
        \item Energy efficiency
        \item High maximum joint torque
        \item High maximum joint velocity
        \item Small size and weight
        \item Robustness to impacts, etc.
        \item Large range of motion
        \item Low price
        \item User-friendly
    \end{itemize}
\end{whitebox}

\begin{whitebox}{\textbf{ACTUATION PRINCIPLES}}
    \begin{itemize}
    \item High-geared motor with torque sensor
    \begin{itemize}
        \itemPro Very compact
        \itemPro Motor can be operated at high speed
        \itemCon High reflected inertia
        \itemCon Low gearbox efficiency
        \itemCon Impact loads can destroy the gear
    \end{itemize}
    \item High-geared motor with serial spring\\ (series elastic actuator)
    \begin{itemize}
        \itemPro Very compact
        \itemPro Precise torque regulation
        \itemPro Spring decouples actuator and link inertia (robustness as motor inertia is not seen during impact)
        \itemPro Additional spring dynamics (temporary energy storage and power/speed amplification)
        \itemCon Low control bandwidth
    \end{itemize}
    \begin{center}
        \includegraphics[width=0.6\textwidth]{media/Series_Elastic_Actuator.png}
    \end{center}
    \item Low-geared high-torque motor (pseudo direct drive)
    \begin{itemize}
        \itemPro Low reflected inertia due to low gear ratio (impact robust, high speed and power)
        \itemPro High bandwidth current control (force control)
        \itemCon Relatively large (hard to integrate)
    \end{itemize}
    %\item Inertia $I_\mathrm{total}=I_\mathrm{link}+I_\mathrm{motor}r^2$, where $I_\mathrm{link}$, $I_\mathrm{motor}$, $r$ are the output link inertia, motor inertia, and transmission ratio, respectively
    \item Hydraulic actuation
    \begin{itemize}
        \itemPro High force at small size/weight
        \itemPro Very rugged
        \itemPro Pressure sensor provides direct force feedback
        \itemCon Onboard pump required
        \itemCon Hard to downscale
        \itemCon Energetically inefficient
        \itemCon Can leak
    \end{itemize}
    \item Pneumatic muscle actuators
    \begin{itemize}
        \itemPro Lightweight
        \itemPro High maximum contraction force
        \itemCon Often with off-board pump
        \itemCon Works only in contraction
        \itemCon Nonlinear contraction-force-pressure characteristics
        \itemCon Difficult to control
        \itemCon Can be quite loud
    \end{itemize}
    \end{itemize}
\end{whitebox}

\begin{whitebox}{\textbf{OTHER ACTUATION TYPES}}
    \begin{itemize}
        \item New, unconventional actuator types
        \begin{itemize}
            \item Shape memory alloy (SMA)
            \item Electro-Active Polymer (EAP)
            \item Piezo-electric
        \end{itemize}
        \item Open issues
        \begin{itemize}
            \item Low output force levels
            \item Low displacement (strain)
            \item Need kV power supplies
            \item Low control bandwidth
        \end{itemize}
    \end{itemize}
\end{whitebox}

\begin{whitebox}{\textbf{STATIC VS. DYNAMICS STABILITY}}
    \begin{itemize}
        \item Statically stable
        \begin{itemize}
            \item Bodyweight supported by at least three legs
            \item Robot will not fall if all joints stop instantaneously
            \item Safe, slow and inefficient
        \end{itemize}
        \item Dynamic walking
        \begin{itemize}
            \item Robot will fall if not continuously moving
            \item Less than three legs can be in ground contact
            \item Fast, efficient and demanding for actuation/control
        \end{itemize}
    \end{itemize}
\end{whitebox}

\begin{whitebox}{\textbf{KINEMATIC CONTROL}}
    \begin{itemize}
        \item High-gain joint position trajectory tracking

        \tikzstyle{block} = [draw, rectangle, minimum height=3em, minimum width=6em]
        \tikzstyle{sum} = [draw, circle, node distance=1cm]
        \tikzstyle{input} = [coordinate]
        \tikzstyle{output} = [coordinate]
        \tikzstyle{pinstyle} = [pin edge={to-,thin,black}]

        \begin{center}
            \scalebox{0.75}{
            \begin{tikzpicture}[auto, node distance=2cm,>=latex']
                % We start by placing the blocks
                \node [block, align=center] (planner) {\footnotesize Motion\\\footnotesize Planner};
                \node [input, name=input] {};
                \node [sum, right of=input, node distance=2.5cm] (sum) {};
                \node [block, right of=sum, align=center] (controller) {\footnotesize PD/PID position\\\footnotesize controller};
                \node [block, right of=controller, node distance=3cm, align=center] (dynamics) {\footnotesize Dynamics};
                % We draw an edge between the controller and system block to 
                % calculate the coordinate u. We need it to place the measurement block. 
                \node [output, right of=dynamics, node distance=2.5cm] (output) {};
                \node [block, below of=controller, align=center, node distance=1.5cm] (measurements) {\footnotesize Sensors};
                % Once the nodes are placed, connecting them is easy. 
                \draw [->] (planner) -- node {$\bm{q}^*,\dot{\bm{q}}^*$} (sum);
                \draw [->] (sum) -- node {} (controller);
                \draw [->] (controller) -- node {$\bm{\tau}$} (dynamics);
                \draw [->] (dynamics) -- node [name=y] {$\bm{q},\dot{\bm{q}},\ddot{\bm{q}}$} (output);
                \draw [->] (y) |- (measurements);
                \draw [->] (measurements) -| node[pos=0.99] {$-$} node [near end] {$\bm{q},\dot{\bm{q}}$} (sum);
            \end{tikzpicture}
            }
        \end{center}
        
        \begin{itemize}
            \itemCon Performs poorly on unknown terrain
        \end{itemize}
    \end{itemize}
\end{whitebox}

\begin{whitebox}{\textbf{JOINT SPACE INVERSE DYNAMICS}}
    \begin{itemize}
        \item Low-gain joint control with model compensation
        
        \tikzstyle{block} = [draw, rectangle, minimum height=3em, minimum width=6em]
        \tikzstyle{sum} = [draw, circle, node distance=1cm]
        \tikzstyle{input} = [coordinate]
        \tikzstyle{output} = [coordinate]
        \tikzstyle{pinstyle} = [pin edge={to-,thin,black}]

        \begin{center}
            \scalebox{0.7}{
            \begin{tikzpicture}[auto, node distance=2cm,>=latex']
                % We start by placing the blocks
                \node [block, align=center] (planner) {\footnotesize Motion\\\footnotesize Planner};
                \node [input, right of =planner] (input) {};
                \node [sum, right of=input] (sum) {};
                \node [block, right of=sum, align=center] (controller) {\footnotesize PD/PID position\\\footnotesize controller};
                \node [block, above of=controller, align=center, node distance=1.5cm] (invdynamics) {\footnotesize Inverse\\\footnotesize dynamics};
                \node [sum, right of=controller,node distance=2cm] (sum_ff) {};
                \node [block, right of=sum_ff, align=center] (dynamics) {\footnotesize Dynamics};
                % We draw an edge between the controller and system block to 
                % calculate the coordinate u. We need it to place the measurement block. 
                \node [output, right of=dynamics, node distance=2.5cm] (output) {};
                \node [block, below of=controller, align=center, node distance=1.5cm] (measurements) {\footnotesize Sensors};
                % Once the nodes are placed, connecting them is easy. 
                \draw [->] (planner) -- node [name=u, below] {$\bm{q}^*,\dot{\bm{q}}^*$} (sum);
                \draw [->] (sum) -- node {} (controller);
                \draw [->] (controller) -- node {$\bm{\tau}_\mathrm{fb}$} (sum_ff);
                \draw [->] (sum_ff) -- node {$\bm{\tau}$} (dynamics);
                \draw [->] (dynamics) -- node [name=y] {$\bm{q},\dot{\bm{q}},\ddot{\bm{q}}$} (output);
                \draw [->] (u) |- (invdynamics);
                \draw [->] (invdynamics) -| node [near end] {$\bm{\tau}_\mathrm{ff}$} (sum_ff);
                \draw [->] (y) |- (measurements);
                \draw [->] (measurements) -| node[pos=0.99] {$-$} node [near end] {$\bm{q},\dot{\bm{q}}$} (sum);
            \end{tikzpicture}
            }
        \end{center}
    \end{itemize}
\end{whitebox}

\begin{whitebox}{\textbf{SUPPORT CONSISTENT INVERSE DYNAMICS}}
    \begin{itemize}
        \item Joint acceleration from multiple objectives
        \begin{itemize}
            \item Track the swing leg (point $F$ on foot)
            \begin{align*}
                &_\mathcal{I}\ddot{\bm{r}}_{IF}=\bm{J}_{F_P}\ddot{\bm{q}}+\dot{\bm{J}}_{F_P}\dot{\bm{q}}\\
                &_\mathcal{I}\ddot{\bm{r}}_{IF}^*=\bm{k}_p(_\mathcal{I}\bm{r}_{IF}^*-\ _\mathcal{I}\bm{r}_{IF})+\bm{k}_d(\dot{\bm{r}}_{IF}^*-\dot{\bm{r}}_{IF})
            \end{align*}
            \item Move the base
            \begin{align*}
                &\dot{\bm{w}}_B=\bm{J}_B\ddot{\bm{q}}+\dot{\bm{J}}_B\dot{\bm{q}}\\
                &\dot{\bm{w}}_B^*=\bm{k}_p\left(
                \begin{bmatrix}
                    \bm{r}^*\\
                    \bm{\varphi}^*
                \end{bmatrix}-
                \begin{bmatrix}
                    \bm{r}\\
                    \bm{\varphi}
                \end{bmatrix}\right)+k_d(\bm{w}^*-\bm{w})
            \end{align*}
            \item Ensure contact constraint
            \begin{align*}
                \ddot{\bm{r}}_c=\bm{J}_{c_P}\ddot{\bm{q}}+\dot{\bm{J}}_{c_P}\dot{\bm{q}}=\bm{0}
            \end{align*}
        \end{itemize}
        \item Above tasks impose $3+6+9=18$ constraints $\implies$ fully define (18 DOF) system  motion
        \item Inverse dynamics control
        \begin{align*}
            &\bm{\tau}^*=\left(\hat{\bm{N}}_c^\top\bm{S}^\top\right)^\dagger\hat{\bm{N}}_c^\top\left(\hat{\bm{M}}\dot{\bm{u}}^*+\hat{\bm{b}}+\hat{\bm{g}}\right)\\
            &\ddot{\bm{q}}^*=
            \begin{bmatrix}
                \bm{J}_{F_P}\\
                \bm{J}_B\\
                \bm{J}_{c_P}
            \end{bmatrix}^{-1}
            \begin{bmatrix}
                _\mathcal{I}\ddot{\bm{r}}_{IF}^*\\
                \dot{\bm{w}}_B^*\\
                \bm{0}
            \end{bmatrix}-
            \begin{bmatrix}
                \dot{\bm{J}}_{F_P}\\
                \dot{\bm{J}}_B\\
                \dot{\bm{J}}_{c_P}
            \end{bmatrix}\dot{\bm{q}}\to\dot{\bm{u}}^*
        \end{align*}
        \item Alternative: task-space inverse dynamics control (direcly regulating in task space as sequential QP)
    \end{itemize}
\end{whitebox}

\begin{whitebox}{\textbf{LOCOMOTION AS OPTIMIZATION PROBLEM}}
    \begin{itemize}
        \item Inverse dynamics as constrained, prioritized optimization
        \begin{itemize}
            \item Step 1: move base
            \begin{align*}
                \arg\min_{\dot{\bm{q}}}||\dot{\bm{w}}_B^*(t)-\bm{J}_B\ddot{\bm{q}}-\dot{\bm{J}}_B\dot{\bm{q}}||\quad\mathrm{\ subject\ to\ }
            \end{align*}
            \begin{center}
                \begin{tabularx}{\columnwidth}{ll}
                    $\bm{M}\ddot{\bm{q}}+\bm{b}+\bm{g}+\bm{J}_c^\top\bm{F}_c=\bm{S}^\top\bm{\tau}$ & EoM\\
                    \addlinespace[2pt]
                    $\bm{J}_{c_P}\ddot{\bm{q}}+\dot{\bm{J}}_{c_P}\dot{\bm{q}}=\bm{0}$ & Contact constraint\\
                    \addlinespace[2pt]
                    $\bm{F}_{c,n_i}>F_{n,\mathrm{min}}$ & Normal contact force\\
                    \addlinespace[2pt]
                    $\mu\bm{F}_{c,n_i}>||\bm{F}_{c,t_i}||_2$ & Friction cone\\
                \end{tabularx}
            \end{center}
            
            \item Step 2: move swing leg
            \begin{align*}
                \arg\min_{\dot{\bm{q}}}||_\mathcal{I}\ddot{\bm{r}}_{IF}^*(t)-\bm{J}_{F_P}\ddot{\bm{q}}-\dot{\bm{J}}_{F_P}\dot{\bm{q}}||\quad\mathrm{\ subject\ to\ }
            \end{align*}
            \begin{center}
                \begin{tabularx}{\columnwidth}{ll}
                    $\dot{\bm{w}}_B^*(t)-\bm{J}_B\ddot{\bm{q}}-\dot{\bm{J}}_B\dot{\bm{q}}=c_1$ & Higher prio unaffected\\
                    \addlinespace[2pt]
                    $\bm{M}\ddot{\bm{q}}+\bm{b}+\bm{g}+\bm{J}_c^\top\bm{F}_c=\bm{S}^\top\bm{\tau}$ & EoM\\
                    \addlinespace[2pt]
                    $\bm{F}_{c,n_i}>F_{n,\mathrm{min}}$ & Normal contact force\\
                    \addlinespace[2pt]
                    $\mu\bm{F}_{c,n_i}>||\bm{F}_{c,t_i}||_2$ & Friction cone\\
                \end{tabularx}
            \end{center}            
            \item Final step: minimize e.g. torque $\bm{\tau}$ or tangential contact forces $\bm{F}_{c,t_i}$ such that all other tasks are still fulfilled
        \end{itemize}
    \end{itemize}
\end{whitebox}

\begin{whitebox}{\textbf{STABILITY ANALYSIS THROUGH LIMIT CYCLES}}
    \begin{itemize}
        \item Poincaré map $\bm{x}_{k+1}=P(\bm{x}_k)$
        \item Fix point characterized by $\bm{x}^*=P(\bm{x}^*)$
        \item Linearization of mapping $\Delta\bm{x}_{k+1}=\frac{\partial P}{\partial\bm{x}}\Delta\bm{x}_k=\bm{\Phi}\Delta\bm{x}_k$
        \item The system is stable iff all eigenvalues $\lambda_i(\bm{\Phi})<1$
    \end{itemize}
\end{whitebox}